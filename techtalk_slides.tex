\documentclass[10pt, compress]{beamer}

\usetheme{m}

\usepackage{booktabs}
\usepackage[scale=2]{ccicons}
\usepackage{minted}
% \usepackage{hyperref}
\usepackage{multicol}
\usepackage{gb4e}
\resetcounteronoverlays{exx}

\usemintedstyle{trac}

\title{Tackling Natural Language Generation Challenges at Narrative Science}
\subtitle{}
\date{Oct 19, 2017}
\author{Mike Pham \& Clayton Norris}
\institute{Narrative Science}

\begin{document}

\maketitle

% Company Overview
% Platform overview
% [Transition into main topic]
%   Machine learning is not a feature
% Deep dive on problems:
%   Verb conjugation
%       Can you machine learn verb forms? Talk to John Goldsmith
%       Rule based approaches kind of work, but you keep having to add exceptions for irregular verbs
%           This varies language to language
%       Really it’s not that hard to just enumerate all verbs with their forms, there aren’t that many verbs that people use
%   Pronouns
%       Take a passage of text with all specific references, ask audience: how do we get it to be less redundant?
%           Maybe use a real passage and replace refs with specific?
%       Go through some rule-based approaches, show where that either gets hairy or doesn’t work
%       “This is a well studied phenomenon called saliency”
%   Sentence selection
%       You obviously can’t enumerate all good sentences
%       Rule systems are just really inextensible
%       This is really hard to pin down, finally a good problem to apply machine learning to
% Conclusions
% Talk about internships now or at the beginning?
% Q&A

\section{Overview}
\begin{frame}{What is Quill?}
    Quill is an \alert{Advanced Natural Language Generation (NLG)} platform

    \begin{description}
        \item[NLG] A form of artificial intelligence (AI) that automatically produces language from structured data. 
        \item[intent-driven] Advanced NLG uses \alert{intent}, or what you want to know, as its guide from the very beginning.
    \end{description}
\end{frame}

\begin{frame}{How is this different than other NLG?}
	So what?
	\pause

	\begin{itemize}
		\item How is this different than Amazon sending me a templated email receipt of my recent purchases?	\pause
		\item What about all those neural nets generating facebook posts that sound eeriely like my previous posts?	\pause
	\end{itemize}
\end{frame}

\section{Verb Conjugation}
\begin{frame}{Verb Inflection}
	\begin{itemize}
		\item A single verb can have various {\bf word forms}:
	\end{itemize}

	\begin{exe}
		\ex {\sc create}	\begin{xlist}
			\ex\label{createInfl} \emph{create, creates, created, creating}
			\ex\label{createDer} \emph{creator, creation, creative, creatively}
		\end{xlist}
	\end{exe}

	\begin{itemize}
		\item (\ref{createInfl}) is an example of {\bf inflectional morphology}
		\begin{itemize}
			\item expresses grammatical features
			\item (usually) doesn't change basic meaning or part of speech
		\end{itemize}
	\end{itemize}
\end{frame}

\begin{frame}{Grammatical features}
	\begin{itemize}
		\item \textbf{Grammatical features} are properties that the grammar of any language tracks and manifests
		\item Some features that English is sensitive to:
		\begin{itemize}
			\item \textbf{number}: \emph{dog, dogs}
			\item \textbf{tense}: \emph{create, created}
			\item \textbf{gender}: \emph{he, she}
			\item \textbf{person}: \emph{we, yall, they}
			\item \textbf{mass/count}: \emph{3 books, *3 bloods}
			\item \textbf{case}: \emph{I, me, my, mine}
		\end{itemize}
	\end{itemize}
\end{frame}

\begin{frame}{Inflectional Paradigms}
	\begin{itemize}
		\item Word forms can track multiple features at once
		\item This can be tracked within an \bf{inflectional paradigm}
	\end{itemize}

	\begin{table}
		{\sc create} \\
		\begin{multicols}{2}

			{\bf Present}	\\
			\begin{tabular}{|r|cc|}
				\toprule
					&	singular	&	plural	\\
				\midrule
				1	&	create 	&	create 	\\	
				% \midrule
				2	&	create 	&	create 	\\
				% \midrule
				3	&	creates &	create 	\\
				\bottomrule
			\end{tabular}

			{\bf Past}	\\
			\begin{tabular}{|r|cc|}
				\toprule
					&	singular	&	plural	\\
				\midrule
				1	&	created 	&	created 	\\	
				% \midrule
				2	&	created 	&	created 	\\
				% \midrule
				3	&	created &	created 	\\
				\bottomrule
			\end{tabular}
		\end{multicols}
	\end{table}
	\pause
	\begin{itemize}
		\item Only 3rd person singular is different -- this looks easy!
		\begin{itemize}
			\item Just add \emph{-s} to the 3.sg present form and \emph{-d} to all past forms!
		\end{itemize}
	\end{itemize}
\end{frame}


\begin{frame}{Irregulars}
	Unfortunately, we all know there are {\bf irregular verbs} in English

	\begin{table}
		{\sc be} \\
		\begin{multicols}{2}

			{\bf Present}	\\
			\begin{tabular}{|r|cc|}
				\toprule
					&	singular	&	plural	\\
				\midrule
				1	&	am 	&	are 	\\	
				% \midrule
				2	&	are 	&	are 	\\
				% \midrule
				3	&	is &	are 	\\
				\bottomrule
			\end{tabular}

			{\bf Past}	\\
			\begin{tabular}{|r|cc|}
				\toprule
					&	singular	&	plural	\\
				\midrule
				1	&	was 	&	were 	\\	
				% \midrule
				2	&	were 	&	were 	\\
				% \midrule
				3	&	was &	were 	\\
				\bottomrule
			\end{tabular}
		\end{multicols}
	\end{table}

	\pause

	\begin{itemize}
		\item Darn, how do we get \emph{am} or \emph{was} from \emph{be}?
	\end{itemize}

\end{frame}


\section{Pronouns}


\section{Sentence Selection}
\begin{frame}{Grammaticality vs Style}
    \begin{description}
        \item[Grammaticality:] only grammatical and accurate sentences should be {\bf generated}
        \item[Sentence selection:] the stylistically best sentence from the set of grammatical candidate sentences should be {\bf selected} \pause
    \end{description}
    \begin{itemize}
        \item but what determines a stylistically `good` sentence?
    \end{itemize}
    
\end{frame}

\begin{frame}{Subjective axes of `goodness`}
    \begin{itemize}
        \item Most native speakers will agree when a sentence is grammatical
        \item But style is vague and elusive, varying from person to person   \pause
        \item Which do think is the best sentence?
    \end{itemize}
    
    \begin{exe}
        \ex Aaron Young generated \$3M in revenue in 2016.
        \ex Aaron Young's revenue was \$3M in 2016.
        \ex Revenue for Aaron Young was \$3M in 2016.
        \ex In 2016, Aaron Young generated \$3M in revenue.
        \ex Aaron Young's 2016 generated revenue was \$3M.
    \end{exe}
    
\end{frame}


\section{Conclusion}





% \begin{frame}[fragile]
%   \frametitle{mtheme}

%   The \emph{mtheme} is a Beamer theme with minimal visual noise inspired by the
%   \href{https://github.com/hsrmbeamertheme/hsrmbeamertheme}{\textsc{hsrm} Beamer
%   Theme} by Benjamin Weiss.

%   Enable the theme by loading

%   \begin{minted}[fontsize=\small]{latex}
%     \documentclass{beamer}
%     \usetheme{m}
%   \end{minted}

%   Note, that you have to have Mozilla's \emph{Fira Sans} font and XeTeX
%   installed to enjoy this wonderful typography.
% \end{frame}

% \begin{frame}[fragile]
%   \frametitle{Sections}
%   Sections group slides of the same topic

%   \begin{minted}[fontsize=\small]{latex}
%     \section{Elements}
%   \end{minted}

%   for which the \emph{mtheme} provides a nice progress indicator \ldots
% \end{frame}

\section{Elements}

\begin{frame}[fragile]
  \frametitle{Typography}
      \begin{minted}[fontsize=\small]{latex}
The theme provides sensible defaults to \emph{emphasis}
text, \alert{accent} parts or show \textbf{bold} results.
      \end{minted}

  \begin{center}becomes\end{center}

  The theme provides sensible defaults to \emph{emphasis} text,
  \alert{accent} parts or show \textbf{bold} results.
\end{frame}

% \begin{frame}{Lists}
%   \begin{columns}[onlytextwidth]
%     \column{0.5\textwidth}
%       Items
%       \begin{itemize}
%         \item Milk \item Eggs \item Potatos
%       \end{itemize}

%     \column{0.5\textwidth}
%       Enumerations
%       \begin{enumerate}
%         \item First, \item Second and \item Last.
%       \end{enumerate}
%   \end{columns}
% % \end{frame}

% \begin{frame}{Descriptions}
%   \begin{description}
%     \item[PowerPoint] Meeh.
%     \item[Beamer] Yeeeha.
%   \end{description}
% \end{frame}

% \begin{frame}{Animation}
%   \begin{itemize}[<+- | alert@+>]
%     \item \alert<4>{This is\only<4>{ really} important}
%     \item Now this
%     \item And now this
%   \end{itemize}
% \end{frame}

% \begin{frame}{Figures}
%   \begin{figure}
%     \newcounter{density}
%     \setcounter{density}{20}
%     \begin{tikzpicture}
%       \def\couleur{mLightBrown}
%       \path[coordinate] (0,0)  coordinate(A)
%                   ++( 90:5cm) coordinate(B)
%                   ++(0:5cm) coordinate(C)
%                   ++(-90:5cm) coordinate(D);
%       \draw[fill=\couleur!\thedensity] (A) -- (B) -- (C) --(D) -- cycle;
%       \foreach \x in {1,...,40}{%
%           \pgfmathsetcounter{density}{\thedensity+20}
%           \setcounter{density}{\thedensity}
%           \path[coordinate] coordinate(X) at (A){};
%           \path[coordinate] (A) -- (B) coordinate[pos=.10](A)
%                               -- (C) coordinate[pos=.10](B)
%                               -- (D) coordinate[pos=.10](C)
%                               -- (X) coordinate[pos=.10](D);
%           \draw[fill=\couleur!\thedensity] (A)--(B)--(C)-- (D) -- cycle;
%       }
%     \end{tikzpicture}
%     \caption{Rotated square from
%     \href{http://www.texample.net/tikz/examples/rotated-polygons/}{texample.net}.}
%   \end{figure}
% \end{frame}

% \begin{frame}{Tables}
%   \begin{table}
%     \caption{Largest cities in the world (source: Wikipedia)}
%     \begin{tabular}{lr}
%       \toprule
%       City & Population\\
%       \midrule
%       Mexico City & 20,116,842\\
%       Shanghai & 19,210,000\\
%       Peking & 15,796,450\\
%       Istanbul & 14,160,467\\
%       \bottomrule
%     \end{tabular}
%   \end{table}
% \end{frame}
% \begin{frame}{Blocks}

%   \begin{block}{This is a block title}
%     This is soothing.
%   \end{block}

% \end{frame}
% \begin{frame}{Math}
%   \begin{equation*}
%     e = \lim_{n\to \infty} \left(1 + \frac{1}{n}\right)^n
%   \end{equation*}
% \end{frame}
% \begin{frame}{Quotes}
%   \begin{quote}
%     Veni, Vidi, Vici
%   \end{quote}
% \end{frame}

\plain{Dark background}{\vspace{-2em}\begin{center}\includegraphics[width=\textwidth]{images/valley.jpg}\end{center}}

\section{Conclusion}

\begin{frame}{Summary}

  Get the source of this theme and the demo presentation from

  \begin{center}\url{github.com/matze/mtheme}\end{center}

  The theme \emph{itself} is licensed under a
  \href{http://creativecommons.org/licenses/by-sa/4.0/}{Creative Commons
  Attribution-ShareAlike 4.0 International License}.

  \begin{center}\ccbysa\end{center}

\end{frame}

\plain{}{Questions?}

\end{document}
